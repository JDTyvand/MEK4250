\documentclass[a4paper,english,12pt,twoside]{article}
\usepackage[english]{babel}
\usepackage[utf8]{inputenc}
\usepackage[T1]{fontenc}
\usepackage[english]{babel}
\usepackage{epsfig}
\usepackage{graphicx}
\usepackage{amsmath}
\usepackage{pstricks}
\usepackage{subcaption}
\usepackage{booktabs}
\usepackage{float}
\usepackage{gensymb}
\usepackage{preamble}
\usepackage{IEEEtrantools}
\restylefloat{table}
\renewcommand{\arraystretch}{1.5}
 \newcommand{\tab}{\hspace*{2em}}


\date{\today}
\title{Mandatory Exercise 2}
\author{Jørgen D. Tyvand}

\begin{document}
\maketitle
\newpage

\section*{\underline{Exercise 1}}

\subsection*{\underline{7.1}}
We are to prove that the following conditions are satisfied for $V_h = H^1_0$ and $Q_h = L^2$
\begin{equation}\tag{7.14}
	a(u_h,v_h) \leq C_1|||u_h||_{V_h}|||v_h||_{V_h},\tab\forall\, u_h, v_h \in V_h
\end{equation}
\begin{equation}\tag{7.15}
	b(u_h,q_h) \leq C_2|||u_h||_{V_h}|||q_h||_{Q_h},\tab\forall\, u_h \in V_h, q_h \in Q_h
\end{equation}
\begin{equation}\tag{7.16}
	a(u_h,u_h) \geq C_3||u_h||^2_{V_h},\tab \forall\, q_h \in Q_h
\end{equation}

First we look at condition (7.14)
\begin{IEEEeqnarray}{rCl}
	a(u_h,v_h) & = & \int_{\Omega}\nabla u_h : \nabla v_h\, dx = \langle\nabla u \,, \nabla v \rangle \leq |\langle\nabla u \,, \nabla v \rangle|\\
	& \leq & ||\nabla u||_{L^2} \cdot ||\nabla v||_{L^2} 
\end{IEEEeqnarray}

Next we look at condition (7.15)
\begin{IEEEeqnarray}{rCl}
	b(u_h, q_h) & = & \int_{\Omega}\nabla \cdot u_h\, q\, dx \leq  \sqrt{\int_{\Omega} (\nabla \cdot u_h\, q)^2\, dx} \leq ||\nabla \cdot u_h\, q||_{L^2}\\
	& \leq & ||q||_{L^2}||\nabla \cdot u_h||_{L^2} \geq \frac{1}{2}(|u_h|^2_{H^1} + C||u_h||||^2_{L^2})\IEEEnonumber\\
	& = & C_3||u_h||^2_{H^1}\IEEEnonumber
\end{IEEEeqnarray}

Finally we look at condition (7.16)
\begin{IEEEeqnarray}{rCl}
	a(u_h,u_h) & = & \int_{\Omega}\nabla u_h : \nabla u_h\, dx = \int_{\Omega} (\nabla u_h)^2\, dx = |u_h|^2_{H^1}\\
	& = & \frac{1}{2}(|u_h|^2_{H^1} + |u_h|^2_{H^1}) \geq \frac{1}{2}(|u_h|^2_{H^1} + C||u_h||^2_{L^2})\IEEEnonumber\\
	& = & C_3||u_h||^2_{H^1}\IEEEnonumber
\end{IEEEeqnarray}


\subsection*{\underline{7.6}}

We are to implement the Stokes problem using $u = (sin(\pi y), cos(\pi x)),$, $p = sin(2\pi x)$ and $f = - \Delta u - \nabla p$, and test whether the approximation\\
\\
$||u - u_h||_1 + ||p - p_h||_0 \leq Ch^k||u||_{k+1} + Dk^{l+1}||p||_{l+1}$\\
\\
where k and l are the polynomial degree of velocity and pressure.\\
\\
The following program checks if the approximation holds
\begin{lstlisting}[style=python, basicstyle = \tiny]
from dolfin import *
import matplotlib.pyplot as plt
import numpy as np
import sys
set_log_active(False)

plt.figure(1)
plt.figure(2)
for i in [[4,3], [4,2], [3,2], [3,1]]:
	print('P%d-P%d' %(i[0],i[1]))
	hvals = []
	UE = []
	PE = []
	for N in [2, 4, 8, 16, 32, 64]:
		h = 1./N
		hvals.append(h)
		mesh = UnitSquareMesh(N,N)
		V = VectorFunctionSpace(mesh, 'Lagrange', i[0])
		Q = FunctionSpace(mesh, 'Lagrange', i[1])

		W = MixedFunctionSpace([V, Q])

		u, p = TrialFunctions(W)
		v, q = TestFunctions(W)

		f = Expression(("pow(pi,2)*sin(pi*x[1]) - 2*pi*cos(2*pi*x[0])",\
				 		"pow(pi,2)*cos(pi*x[0])"))

		u_ex = Expression(("sin(pi*x[1])","cos(pi*x[0])"))

		p_ex = Expression("sin(2*pi*x[0])")

		bc_u = DirichletBC(W.sub(0), u_ex, "on_boundary")
		bc_p = DirichletBC(W.sub(1), p_ex, "on_boundary")
		bc = [bc_u, bc_p]

		a = inner(grad(u), grad(v))*dx + div(u)*q*dx + div(v)*p*dx
		L = inner(f, v)*dx

		up_ = Function(W)
		A, b = assemble_system(a, L, bc)
		solve(a == L, up_, bc)

		u_, p_ = up_.split(True)

		Uerr = errornorm(u_ex,u_, norm_type='h1', degree_rise=1)
		UE.append(Uerr)
		Perr = errornorm(p_ex, p_, norm_type='l2', degree_rise=1)
		PE.append(Perr)

	Ur = []
	Pr = []
	for j in range(1,len(UE)):
		Uconv = np.log(UE[j-1]/UE[j])/np.log(hvals[j-1]/hvals[j])
		Ur.append(Uconv)
		Pconv = np.log(PE[j-1]/PE[j])/np.log(hvals[j-1]/hvals[j])
		Pr.append(Pconv)
		print ('h = %f    r_U = %f    r_P = %f' % (hvals[j], Uconv, Pconv))
	plt.figure(1)
	label = 'P%d-P%d' %(i[0],i[1])
	plt.loglog(hvals,UE, label=label)
	plt.figure(2)
	label = 'P%d-P%d' %(i[0],i[1])
	plt.loglog(hvals,PE, label=label)
	print('\n')

	
plt.figure(1)
plt.legend(loc='upper left')
plt.savefig('Velocity_convergence_P%d-P%d.png' % (i[0],i[1]))

plt.figure(2)
plt.legend(loc='upper left')
plt.savefig('Pressure_convergence_P%d-P%d.png' % (i[0],i[1]))

\end{lstlisting}

\subsection*{\underline{7.7}}
\newpage
\section*{\underline{Exercise 2}}

We are looking ate the topic of 'locking', and are considering the following problem on the unit square domain $\Omega = (0,1)^2:$
\begin{IEEEeqnarray}{rCl}
	-\mu\Delta \vec{u} - \lambda\nabla\nabla\cdot \vec{u} & = & f \text{ in } \Omega\\
	\vec{u} & = & \vec{u_e} \text{ on } \partial\Omega
\end{IEEEeqnarray}
where $\vec{u_e} = \left(\pdi{\phi}{y}, - \pdi{\phi}{x}\right)$, $\phi = sin(\pi x y)$ and $\nabla\cdot\vec{u_e} = 0$.

\subsection*{a)}

We are to derive and expression for $f$. We do this by looking at equation (1) using the exact solution $\vec{u_e}$. Seeing as $\Delta\cdot\vec{u_e} = 0$, equation(1) reduces to
\begin{IEEEeqnarray}{rCl}
	-\mu\Delta \vec{u_e} & = & f
\end{IEEEeqnarray}
We look at the Laplacian term
\begin{IEEEeqnarray}{rCl}
	\Delta \vec{u_e}  = \nabla^2\vec{u_e} & =  & (\nabla^2 u, \nabla^2 v) = \left(  \ppder{u}{x} + \ppder{u}{y} , \ppder{v}{x} + \ppder{v}{y}  \right)
\end{IEEEeqnarray}
where $u$ and $v$ are the components of $\vec{u_e}$. We get
\begin{IEEEeqnarray}{rCl}
	 \nabla^2\vec{u_e} & =  & \bigg( \pdi{}{x}[-\pi cos(\pi x y) - \pi^2 xysin(\pi xy)]  + \pdi{}{y}[-\pi^2x^2sin(\pi xy)],\\
&& \pdi{}{x}[\pi^2y^2sin(\pi xy)] + \pdi{}{y}[-\pi cos(\pi xy) + \pi^2 xysin(\pi xy)]\bigg) \IEEEnonumber\\
& =  & \bigg(-\pi^2 y sin(\pi xy) - \pi^2 y sin(\pi xy) - \pi^3 xy^2cos(\pi xy) -\pi^3x^3cos(\pi xy),\\
&& \pi^3y^3cos(\pi xy) + \pi^2xsin(\pi xy) + \pi^2 xsin(\pi xy) + \pi^3x^2ycos(\pi xy)\bigg) \IEEEnonumber\\
& =  & \bigg(-\pi^2[2ysin(\pi xy) + \pi x(x^2 + y^2)cos(\pi xy)],\\
&& \pi^2[2xsin(\pi xy) + \pi y(x^2 + y^2)cos(\pi xy)]\bigg) \IEEEnonumber
\end{IEEEeqnarray}
Inserting this into equation (3) gives us
\begin{IEEEeqnarray}{rCl}
	f & = & \mu\bigg(\pi^2[2ysin(\pi xy) + \pi x(x^2 + y^2)cos(\pi xy)],\\
&& -\pi^2[2xsin(\pi xy) + \pi y(x^2 + y^2)cos(\pi xy)]\bigg) \IEEEnonumber
\end{IEEEeqnarray}

\subsection*{b)}

Next we want to compute the numerical error for $\lambda = 1, 100, 10000$ at $h = 8, 16, 32, 64$ for polynomial order 1 and 2.\\
\\
The following code computes and outputs the numerical error
\begin{lstlisting}[style=python, basicstyle = \tiny]
from dolfin import *
import sys
set_log_active(False)


lvals = [1., 100., 10000.]
nvals = [8,16,32,64]

for i in [0,1]:
	print("P%d Elements\n" % (i+1))
	for l in lvals:
		for N in nvals:

			mesh = UnitSquareMesh(N,N)

			h = 1.0/N

			V = VectorFunctionSpace(mesh, 'Lagrange', i+1)
			V_1 = VectorFunctionSpace(mesh, 'Lagrange', i+1)

			u = TrialFunction(V)
			v = TestFunction(V)


			u_ex = interpolate(Expression(("pi*x[0]*cos(pi*x[0]*x[1])",\
					"-pi*x[1]*cos(pi*x[0]*x[1])")),V_1)
			
			bc = DirichletBC(V, u_ex, "on_boundary")
			bcs = [bc]

			mu = 1.

			f = interpolate(Expression(("mu * pow(pi,2) \
						 * (2 * x[1] * sin(pi*x[0]*x[1])\
						+ pi * x[0] *(pow(x[0],2) + \
						pow(x[1],2))*cos(pi*x[0]*x[1]))"
						,  "-mu * pow(pi,2) *\
						(2 * x[0] * sin(pi*x[0]*x[1]) + pi * x[1] \
						*(pow(x[0],2) + pow(x[1],2)) *\
						cos(pi*x[0]*x[1]))"),	mu=mu),V_1)
			

			a = mu*inner(grad(u),grad(v))*dx + l*inner(div(u),div(v))*dx
			L =	dot(f,v)*dx
			u_ = Function(V)

			solve(a == L, u_, bcs)	
			
			err = errornorm(u_, u_ex, norm_type='l2', degree_rise=1)
			print ('h = %f   lambda = %-5d  error = %e' % (h, l, err))
		print
	print
\end{lstlisting}
\newpage
The output from the program is listed below
\begin{lstlisting}[style=terminal, basicstyle=\small]
P1 Elements

h = 0.125000   lambda = 1      error = 3.665490e-02
h = 0.062500   lambda = 1      error = 9.893346e-03
h = 0.031250   lambda = 1      error = 2.523885e-03
h = 0.015625   lambda = 1      error = 6.342241e-04

h = 0.125000   lambda = 100    error = 2.909305e-01
h = 0.062500   lambda = 100    error = 1.624590e-01
h = 0.031250   lambda = 100    error = 6.039903e-02
h = 0.015625   lambda = 100    error = 1.756415e-02

h = 0.125000   lambda = 10000  error = 4.379122e-01
h = 0.062500   lambda = 10000  error = 4.551903e-01
h = 0.031250   lambda = 10000  error = 4.327456e-01
h = 0.015625   lambda = 10000  error = 3.518862e-01


P2 Elements

h = 0.125000   lambda = 1      error = 6.622818e-04
h = 0.062500   lambda = 1      error = 4.404308e-05
h = 0.031250   lambda = 1      error = 2.810815e-06
h = 0.015625   lambda = 1      error = 1.769528e-07

h = 0.125000   lambda = 100    error = 1.425217e-02
h = 0.062500   lambda = 100    error = 1.477785e-03
h = 0.031250   lambda = 100    error = 1.154063e-04
h = 0.015625   lambda = 100    error = 7.836387e-06

h = 0.125000   lambda = 10000  error = 2.981601e-02
h = 0.062500   lambda = 10000  error = 7.169928e-03
h = 0.031250   lambda = 10000  error = 1.576881e-03
h = 0.015625   lambda = 10000  error = 2.721675e-04
\end{lstlisting}
\end{document}