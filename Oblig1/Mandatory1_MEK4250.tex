\documentclass[a4paper,english,12pt,twoside]{article}
\usepackage[english]{babel}
\usepackage[utf8]{inputenc}
\usepackage[T1]{fontenc}
\usepackage[english]{babel}
\usepackage{epsfig}
\usepackage{graphicx}
\usepackage{amsmath}
\usepackage{pstricks}
\usepackage{subcaption}
\usepackage{booktabs}
\usepackage{float}
\usepackage{gensymb}
\usepackage{preamble}
\restylefloat{table}
\renewcommand{\arraystretch}{1.5}
 \newcommand{\tab}{\hspace*{2em}}


\date{\today}
\title{Mandatory Assignment 1, MEK 4250}
\author{Jørgen D. Tyvand}

\begin{document}
\maketitle
\newpage

\section*{\underline{Exercise 1}}

We are considering the problem\\
\\
$-\Delta u = f$ in $\Omega$\\
\\
$u = 0$ for $x = 0$ and $x = 1$\\
\\
$\pdi{u}{n} = 0$ for $y = 0$ and $y =1$\\
\\
We are asked to compute $f = -\Delta u$, assuming $u = sin(\pi k x)cos(\pi l y)$. This is done as follows\\
\\
$f = -\ppder{u}{x} -\ppder{u}{y} = (k\pi)^2sin(\pi k x)cos(\pi l y) + (l\pi)^2sin(\pi k x)cos(\pi l y)  = ((k\pi)^2 + (l\pi)^2)u$
\subsection*{a)}

I assume that the purpose of "Computing the $H^p$ norm of $u$" suggests finding an analytical expression for $H^p$, rather than an actual computation with a program, as no value is given for p.\\
\\
Looking at the general expression for the $H^p$ norm, we have the following expression\\
\\
$\displaystyle H^p = \left[\int_{\Omega}\sum_{i\leq p}|\nabla^i u|^2 \diff x \right]^ \frac{1}{2}$\\
\\
The term we want to look at is the increasing powers of the gradient/divergence term $\nabla^i u$. This will alternate between being a scalar and a vector, but the term is squared in the above expression, giving a scalar value. We get\\
\\
$\nabla^0 u = u = sin(\pi k x)cos(\pi l y)$\\
\\
$\displaystyle \int_{\Omega}u^2 \diff y \diff x = \int_{\Omega}sin^2(\pi k x)cos^2(\pi l y) \diff y \diff x = \frac{1}{4}$\\
\\
\\
$\nabla^1 u = \left[\pi k cos(\pi k x)cos(\pi l y), -\pi l sin(\pi k x)sin(\pi l y) \right]$\\
\\
$\displaystyle \int_{\Omega}(\nabla u)^2 \diff y \diff x = \int_{\Omega}\left[\pi^2k^2cos^2(\pi k x)cos^2(\pi l y) + \pi^2l^2sin^2(\pi k x)sin^2(\pi l y)\right] \diff y \diff x = \frac{\pi^2(k^2 + l^2)}{4}$\\
\\
\\
$\nabla^2 u = \nabla\cdot\nabla u = -\pi^2 k^2 sin(\pi k x)cos(\pi l y) -\pi^2 l^2 sin(\pi k x)cos(\pi l y) = -\pi^2(k^2 + l^2)u$\\
\\
$\displaystyle \int_{\Omega}(\nabla^2 u)^2 \diff y \diff x = \pi^4(k^2 + l^2)^2\int_{\Omega}u^2 \diff y \diff x = \frac{\pi^4(k^2 + l^2)^2}{4}$\\
\\
\\
$\nabla^3 u = \nabla(\nabla^2u) = \left[-(\pi^3) k(k^2 + l^2) cos(\pi k x)cos(\pi l y), \pi^3 l (k^2 + l^2) sin(\pi k x)sin(\pi l y) \right]$\\
\\
$\displaystyle \int_{\Omega}(\nabla^3 u)^2 \diff y \diff x = \pi^6(k^2 + l^2)^3\int_{\Omega}\left[cos^2(\pi k x)cos^2(\pi l y) + sin^2(\pi k x)sin^2(\pi l y)\right] \diff y \diff x$\\
\\
\\
$\displaystyle \tab\tab\tab\tab = \frac{\pi^6(k^2 + l^2)^3}{4}$\\
\\
\\
$\nabla^4 u = \nabla\cdot\nabla^3u = \pi^4 k^2 (k^2 + l^2) sin(\pi k x)cos(\pi l y)  + \pi^4 l^2 (k^2 + l^2) sin(\pi k x)cos(\pi l y)$\\
\\
\\
$\displaystyle \tab\tab\tab\tab =\pi^4(k^2 + l^2)^2u$\\
\\
\\
$\displaystyle \int_{\Omega}(\nabla^4 u)^2 \diff y \diff x = \pi^8(k^2 + l^2)^4\int_{\Omega}u \diff y \diff x = \frac{ \pi^8(k^2 + l^2)^4}{4}$\\
\\
\\
As we can see, there is a pattern to the value of the integral, and inserted into the expression for the $H^p$ norm, we get\\
\\
$\displaystyle H^p = \frac{1}{2}\left[\sum_{i\leq p}\pi^{2i}(k^2 + l^2)^i\right]^\frac{1}{2}$\\

\newpage

\subsection*{a)}

The program for computing the $L_2$ and $H^1$ errors is listed below\\
\\
\begin{lstlisting}[style=python, basicstyle = \tiny]
from dolfin import *
import numpy as np
set_log_active(False)

k_vals = [1, 10, 100]
l_vals = [1, 10, 100]

for i in [1,2]:
	for h in [8, 16, 32, 64]:
		mesh = UnitSquareMesh(h,h)
		print('h = %d, P%d elements:\n' % (h,i))
		V = FunctionSpace(mesh, 'Lagrange', i)

		u = TrialFunction(V)
		v = TestFunction(V)

		bc0 = DirichletBC(V, Constant(0.0), 'x[0] < DOLFIN_EPS')
		bc1 = DirichletBC(V, Constant(0.0), 'x[0] > 1-DOLFIN_EPS')
		bcs = [bc0, bc1]


		L2_errs = np.zeros((3,3))
		H1_errs = np.zeros((3,3))
		for k in range(3):
			for l in range(3):

				f = Expression('pi*pi*(k*k + l*l)*sin(pi*k*x[0])*cos(pi*l*x[1])'\
					,k=k_vals[k], l=l_vals[l])

				a = inner(grad(u),grad(v))*dx
				L = f*v*dx

				u_ = Function(V)

				solve(a == L, u_, bcs)

				u_exp = Expression('sin(pi*k*x[0])*cos(pi*l*x[1])',\
					k=k_vals[k],l=l_vals[l])
				u_ex = interpolate(u_exp,V)

				L2_errs[k][l] = errornorm(u_, u_ex, 'l2', degree_rise=3)
				H1_errs[k][l] = errornorm(u_, u_ex, 'h1', degree_rise=3)
\end{lstlisting}
\newpage
A printout of this program (not included above) gives the following results for the different combinations\\

\begin{lstlisting} [style=terminal, basicstyle= \tiny]
h = 8, P1 elements:

L2 norm:

           	l = 1    	l = 10    	l = 100
k =    1	0.032775	0.679790	191.311299
k =   10	0.679028	0.667057	39.001974
k =  100	193.304690	43.850560	159.356420


H1 norm:

           	l = 1    	l = 10    	l = 100
k =    1	0.436592	16.149851	2760.519420
k =   10	16.403729	26.481453	1062.025335
k =  100	2769.298737	1081.558408	3226.285685


h = 8, P2 elements:

L2 norm:

           	l = 1    	l = 10    	l = 100
k =    1	0.000569	0.329865	289.591947
k =   10	0.331676	0.435611	32.740710
k =  100	289.695710	32.023576	293.246151


H1 norm:

           	l = 1    	l = 10    	l = 100
k =    1	0.033185	9.011149	3779.659405
k =   10	8.910024	19.124524	1121.761657
k =  100	3780.341160	1119.416192	5321.280870


h = 16, P1 elements:

L2 norm:

           	l = 1    	l = 10    	l = 100
k =    1	0.008463	0.245283	262.037348
k =   10	0.240959	0.365538	22.107864
k =  100	262.277492	23.046334	246.861840


H1 norm:

           	l = 1    	l = 10    	l = 100
k =    1	0.218166	9.179273	3505.998736
k =   10	9.097867	17.546447	871.935651
k =  100	3507.090063	873.593192	4686.200990


h = 16, P2 elements:

L2 norm:

           	l = 1    	l = 10    	l = 100
k =    1	0.000069	0.025310	91.898771
k =   10	0.025276	0.089599	8.702181
k =  100	91.923500	9.070227	90.474916


H1 norm:

           	l = 1    	l = 10    	l = 100
k =    1	0.008389	2.207236	1219.798089
k =   10	2.183449	6.920349	386.281221
k =  100	1219.859911	383.828456	1648.496211
\end{lstlisting}
\newpage
\begin{lstlisting} [style=terminal, basicstyle= \tiny]
h = 32, P1 elements:

L2 norm:

           	l = 1    	l = 10    	l = 100
k =    1	0.002133	0.078653	3.079839
k =   10	0.078315	0.178190	2.650246
k =  100	2.931370	2.447102	2.696856


H1 norm:

           	l = 1    	l = 10    	l = 100
k =    1	0.109055	4.623558	281.577212
k =   10	4.605848	10.602375	269.120343
k =  100	281.670147	266.895911	376.364409


h = 32, P2 elements:

L2 norm:

           	l = 1    	l = 10    	l = 100
k =    1	0.000009	0.002889	5.903254
k =   10	0.002879	0.010206	5.120845
k =  100	5.906873	5.124357	4.722304


H1 norm:

           	l = 1    	l = 10    	l = 100
k =    1	0.002106	0.572304	556.318121
k =   10	0.568733	1.977957	520.797098
k =  100	556.632761	521.094856	689.092405


h = 64, P1 elements:

L2 norm:

           	l = 1    	l = 10    	l = 100
k =    1	0.000534	0.020911	4.688718
k =   10	0.020920	0.054904	4.019684
k =  100	4.687891	4.013733	3.588809


H1 norm:

           	l = 1    	l = 10    	l = 100
k =    1	0.054524	2.283389	433.257497
k =   10	2.280790	5.439857	405.332937
k =  100	433.699674	405.623107	540.466888


h = 64, P2 elements:

L2 norm:

           	l = 1    	l = 10    	l = 100
k =    1	0.000001	0.000351	1.803051
k =   10	0.000350	0.001140	1.579982
k =  100	1.802916	1.580365	1.470963


H1 norm:

           	l = 1    	l = 10    	l = 100
k =    1	0.000527	0.144695	186.599439
k =   10	0.144221	0.518416	178.944455
k =  100	186.504757	178.853083	288.597177
\end{lstlisting}

\subsection*{c)}

We start by rewriting the expression for the error estimate (using the first one given in the exercise as our example)\\
\\
$||u-u_h||_1 \leq C_{\alpha} h^{\alpha}$\\
\\
Assigning the constant A to the norm for brevity and taking the logarithm\\
\\
$A \leq C_{\alpha}h^{\alpha}$\\
\\
$ln(A) \leq ln(C_{\alpha}h^{\alpha})$\\
\\
$ln(A) \leq ln(C_{\alpha}) + \alpha ln(h)$\\
\\
This is a linear equation $y = a + bx$ with $ln(C_{\alpha})$ as the intercept and $\alpha$ as the slope.\\
\\
The below program calculatates the values for $\alpha$ and C for both cases, and checks if the error estimate is valid\\
\\
\begin{lstlisting}[style=python, basicstyle = \tiny]
from dolfin import *
import numpy as np
import matplotlib.pyplot as plt
set_log_active(False)

for k in [1, 10, 100]:
	for err_type in ['H1', 'L2']:
		for i in [1,2]:
		
			x = []
			y = []
			for N in [8,16,32,64]:

				mesh = UnitSquareMesh(N,N)

				h = 1.0/N

				V = FunctionSpace(mesh, 'Lagrange', i)

				u = TrialFunction(V)
				v = TestFunction(V)

				bc0 = DirichletBC(V, Constant(0.0), 'x[0] < DOLFIN_EPS')
				bc1 = DirichletBC(V, Constant(0.0), 'x[0] > 1-DOLFIN_EPS')
				bcs = [bc0, bc1]


				f = Expression('pi*pi*(k*k + l*l)*sin(pi*k*x[0])*cos(pi*l*x[1])'\
					,k=k, l=k)

				a = inner(grad(u),grad(v))*dx
				L = f*v*dx

				u_ = Function(V)

				solve(a == L, u_, bcs)

				u_ex = interpolate(Expression('sin(pi*k*x[0])*cos(pi*l*x[1])', \
					k=k, l=k),V)

				A = errornorm(u_ex, u_, err_type, degree_rise=3)

				x.append(ln(h))
				y.append(ln(A))


			x = np.array(x)
			y = np.array(y)

			Arr = np.vstack([x, np.ones(len(x))]).T
			alpha, lnC = np.linalg.lstsq(Arr, y)[0]
			print('%s, k = %3d, P%d elements:' % (err_type, k, i))
			print('alpha/beta = %.3f, C = %9.3f' % (alpha, exp(lnC)))
			#print('alpha/beta = %.3f, C = %9.3f' % (alpha, lnC))
			RHS = exp(lnC)*h**alpha
			print('Errornorm = %.16f, Ch^alpha/beta = %.16f' % (A, RHS))
			if A < RHS:
				print('Error estimate is valid!\n')
			else:
				print('Error estimate is NOT valid!\n')
\end{lstlisting}

\newpage

The values for the two cases and the result of error estimate validations are listed below

\begin{lstlisting} [style=terminal, basicstyle= \tiny]
H1, k =   1, P1 elements:

alpha/beta = 1.000, C =     3.495

N = 8: Errornorm = 0.43659, Ch^alpha/beta = 0.43653
Error estimate is NOT valid for N = 8!

N = 16: Errornorm = 0.21817, Ch^alpha/beta = 0.21820
Error estimate is valid for N = 16!

N = 32: Errornorm = 0.10906, Ch^alpha/beta = 0.10907
Error estimate is valid for N = 32!

N = 64: Errornorm = 0.05452, Ch^alpha/beta = 0.05452
Error estimate is NOT valid for N = 64!



H1, k =   1, P2 elements:

alpha/beta = 1.992, C =     2.096

N = 8: Errornorm = 0.03318, Ch^alpha/beta = 0.03327
Error estimate is valid for N = 8!

N = 16: Errornorm = 0.00839, Ch^alpha/beta = 0.00836
Error estimate is NOT valid for N = 16!

N = 32: Errornorm = 0.00211, Ch^alpha/beta = 0.00210
Error estimate is NOT valid for N = 32!

N = 64: Errornorm = 0.00053, Ch^alpha/beta = 0.00053
Error estimate is valid for N = 64!



L2, k =   1, P1 elements:

alpha/beta = 1.980, C =     2.031

N = 8: Errornorm = 0.03278, Ch^alpha/beta = 0.03305
Error estimate is valid for N = 8!

N = 16: Errornorm = 0.00846, Ch^alpha/beta = 0.00838
Error estimate is NOT valid for N = 16!

N = 32: Errornorm = 0.00213, Ch^alpha/beta = 0.00212
Error estimate is NOT valid for N = 32!

N = 64: Errornorm = 0.00053, Ch^alpha/beta = 0.00054
Error estimate is valid for N = 64!



L2, k =   1, P2 elements:

alpha/beta = 3.015, C =     0.299

N = 8: Errornorm = 0.00057, Ch^alpha/beta = 0.00057
Error estimate is NOT valid for N = 8!

N = 16: Errornorm = 0.00007, Ch^alpha/beta = 0.00007
Error estimate is valid for N = 16!

N = 32: Errornorm = 0.00001, Ch^alpha/beta = 0.00001
Error estimate is valid for N = 32!

N = 64: Errornorm = 0.00000, Ch^alpha/beta = 0.00000
Error estimate is NOT valid for N = 64!

\end{lstlisting}
\newpage
\begin{lstlisting} [style=terminal, basicstyle= \tiny]
H1, k =  10, P1 elements:

alpha/beta = 0.758, C =   135.960

N = 8: Errornorm = 26.48145, Ch^alpha/beta = 28.12922
Error estimate is valid for N = 8!

N = 16: Errornorm = 17.54645, Ch^alpha/beta = 16.63692
Error estimate is NOT valid for N = 16!

N = 32: Errornorm = 10.60237, Ch^alpha/beta = 9.83984
Error estimate is NOT valid for N = 32!

N = 64: Errornorm = 5.43986, Ch^alpha/beta = 5.81973
Error estimate is valid for N = 64!



H1, k =  10, P2 elements:

alpha/beta = 1.742, C =   782.070

N = 8: Errornorm = 19.12452, Ch^alpha/beta = 20.88578
Error estimate is valid for N = 8!

N = 16: Errornorm = 6.92035, Ch^alpha/beta = 6.24289
Error estimate is NOT valid for N = 16!

N = 32: Errornorm = 1.97796, Ch^alpha/beta = 1.86604
Error estimate is NOT valid for N = 32!

N = 64: Errornorm = 0.51842, Ch^alpha/beta = 0.55777
Error estimate is valid for N = 64!



L2, k =  10, P1 elements:

alpha/beta = 1.185, C =     8.891

N = 8: Errornorm = 0.66706, Ch^alpha/beta = 0.75728
Error estimate is valid for N = 8!

N = 16: Errornorm = 0.36554, Ch^alpha/beta = 0.33318
Error estimate is NOT valid for N = 16!

N = 32: Errornorm = 0.17819, Ch^alpha/beta = 0.14659
Error estimate is NOT valid for N = 32!

N = 64: Errornorm = 0.05490, Ch^alpha/beta = 0.06450
Error estimate is valid for N = 64!



L2, k =  10, P2 elements:

alpha/beta = 2.887, C =   211.265

N = 8: Errornorm = 0.43561, Ch^alpha/beta = 0.52215
Error estimate is valid for N = 8!

N = 16: Errornorm = 0.08960, Ch^alpha/beta = 0.07060
Error estimate is NOT valid for N = 16!

N = 32: Errornorm = 0.01021, Ch^alpha/beta = 0.00955
Error estimate is NOT valid for N = 32!

N = 64: Errornorm = 0.00114, Ch^alpha/beta = 0.00129
Error estimate is valid for N = 64!
\end{lstlisting}
\newpage
\begin{lstlisting} [style=terminal, basicstyle= \tiny]
H1, k = 100, P1 elements:

alpha/beta = 1.137, C = 45954.859

N = 8: Errornorm = 3226.28569, Ch^alpha/beta = 4319.43930
Error estimate is valid for N = 8!

N = 16: Errornorm = 4686.20099, Ch^alpha/beta = 1963.93007
Error estimate is NOT valid for N = 16!

N = 32: Errornorm = 376.36441, Ch^alpha/beta = 892.94491
Error estimate is valid for N = 32!

N = 64: Errornorm = 540.46689, Ch^alpha/beta = 405.99745
Error estimate is NOT valid for N = 64!



H1, k = 100, P2 elements:

alpha/beta = 1.387, C = 87018.317

N = 8: Errornorm = 5321.28087, Ch^alpha/beta = 4862.00991
Error estimate is NOT valid for N = 8!

N = 16: Errornorm = 1648.49621, Ch^alpha/beta = 1858.73535
Error estimate is valid for N = 16!

N = 32: Errornorm = 689.09241, Ch^alpha/beta = 710.59031
Error estimate is valid for N = 32!

N = 64: Errornorm = 288.59718, Ch^alpha/beta = 271.65706
Error estimate is NOT valid for N = 64!



L2, k = 100, P1 elements:

alpha/beta = 2.293, C = 31760.852

N = 8: Errornorm = 159.35642, Ch^alpha/beta = 269.60999
Error estimate is valid for N = 8!

N = 16: Errornorm = 246.86184, Ch^alpha/beta = 54.99847
Error estimate is NOT valid for N = 16!

N = 32: Errornorm = 2.69686, Ch^alpha/beta = 11.21929
Error estimate is valid for N = 32!

N = 64: Errornorm = 3.58881, Ch^alpha/beta = 2.28865
Error estimate is NOT valid for N = 64!



L2, k = 100, P2 elements:

alpha/beta = 2.718, C = 99528.778

N = 8: Errornorm = 293.24615, Ch^alpha/beta = 349.59800
Error estimate is valid for N = 8!

N = 16: Errornorm = 90.47492, Ch^alpha/beta = 53.14255
Error estimate is NOT valid for N = 16!

N = 32: Errornorm = 4.72230, Ch^alpha/beta = 8.07822
Error estimate is valid for N = 32!

N = 64: Errornorm = 1.47096, Ch^alpha/beta = 1.22797
Error estimate is NOT valid for N = 64!
\end{lstlisting}

\newpage

\section*{\underline{Exercise 2}}

The problem in this exercise is\\
\\
$\displaystyle -\mu\Delta u + u_x = 0$ in $\Omega$\\
\\
$u = 0$ for $x = 0 \tab\tab u = 1$ for $x = 1$\\
\\
$\displaystyle \pdi{u}{n} = 0$ for $y = 0$ and $y =1$\\

\subsection*{a)}

We wish to derive an expression for the analytical solution. We assume that the solution is a product of two functions dependent on only one parameter, so that we can use separation of variables\\
\\
$u = X(x)Y(x)$\\
\\
Inserting this into the problem equation, we get\\
\\
$\displaystyle -\mu(X''Y'' + XY'') + X'Y = 0$\\
\\
Dividing by $XY$ we get\\
\\
$\displaystyle -\mu\left(\frac{X''}{Y} + \frac{Y''}{Y}\right) + \frac{X'}{X} = 0$\\
\\
We see that we can get two expressions that are only dependent on x or y, so these can be said to be equal to a common constant $\lambda$. We start by looking at the expression for $Y(y)$\\
\\
$\displaystyle \mu\frac{Y''}{Y} = -\lambda^2$\\
\\
$\displaystyle \mu Y''- \lambda Y = 0$\\
\\
$\displaystyle Y''-\frac{\lambda}{\mu}Y = 0$\\
\\
The solution of this equation is $\displaystyle Y(y) = Acos\left(\frac{\lambda}{\sqrt{\mu}}y\right) + Bsin\left(\frac{\lambda}{\sqrt{\mu}}y\right)$\\
\\
\\
The derivative is $\displaystyle Y'(y) = -\frac{A\lambda}{\sqrt{\mu}} sin\left(\frac{\lambda}{\sqrt{\mu}}y\right) + \frac{B\lambda}{\sqrt{\mu}} cos\left(\frac{\lambda}{\sqrt{\mu}}y\right) $\\
\\
\\
Using the boundary conditions, we see that both constants become zero if $\lambda \neq 0$. Therefore $Y(y) = C$, and we can set $Y(y) = 1$ for convenience.\\
\\
Next we look at the equation for X, using the above value $\lambda = 0$\\
\\
$-\mu\frac{X''}{X} + \frac{X'}{X} = 0$\\
\\
$-mu X'' + X' = 0$\\
\\
Using the boundary conditions, and writing out for $u(x)$ seeing that $Y(y) = 1$, we get the solution (similar to the one given in the beginning of chapter 6 of the lecture notes)\\
\\
$\displaystyle u(x) = \frac{e^\frac{x}{\mu} - 1}{e^\frac{1}{\mu} - 1}$

\subsection*{b)}

The following program computes the errors for $mu$ = 1, 0.1, 0.01, 0.001, 0.0001 at N = 8, 16, 32 and 64
\begin{lstlisting} [style=python, basicstyle= \tiny]
from dolfin import *
import numpy as np
set_log_active(False)

muvals = [1., 0.1, 0.01]
nvals = [8, 16, 32, 64]
for i in [1,2]:
	for err_type in ['H1', 'L2']:
		print('	    %s error for P%d elements\n' % (err_type, i))

		errs = np.zeros((3,4))

		print('           	  N = 8  	  N = 16 	  N = 32  	  N = 64')
		ncount = 0
		for N in nvals:
		
			mesh = UnitSquareMesh(N,N)
			V = FunctionSpace(mesh, 'Lagrange', i)
			V_1 = FunctionSpace(mesh, 'Lagrange', i+2)

			u = TrialFunction(V)
			v = TestFunction(V)

			bc0 = DirichletBC(V, Constant(0), 'x[0] < DOLFIN_EPS')
			bc1 = DirichletBC(V, Constant(1), 'x[0] > 1 - DOLFIN_EPS')
			bcs = [bc0, bc1]

			u_ = Function(V)

			f = Constant(0.0)
			
			mucount = 0
			for mu in muvals:
				a = mu*inner(grad(u),grad(v))*dx + u.dx(0)*v*dx
				L = f*v*dx

				solve(a == L, u_, bcs)
				
				u_ex = interpolate(Expression('(exp(x[0]/mu) - \
					1)/(exp(1./mu) - 1)', \
					mu=mu),V_1)

				errs[mucount][ncount] = (errornorm(u_ex,u_, err_type))

				mucount += 1
			ncount += 1

		for j in range(3):
			print('mu = ' + '{:.1e}'.format(muvals[j]) + '	' +  \
				'{:.3e}'.format(errs[j][0]) + \
				'	' + '{:.3e}'.format(errs[j][1]) + '	' + \
				'{:.3e}'.format(errs[j][2]) + \
				'	' + '{:.3e}'.format(errs[j][3]))
		print('\n')



\end{lstlisting}
Below is a printout of the computed errors for $\mu$ = 1, 0.1 and 0.01. The errors for 0.001 and 0.0001 are computed as NaN, so these have not been included here.
\begin{lstlisting} [style=terminal, basicstyle= \tiny]
	    H1 error for P1 elements

           	  N = 8  	  N = 16 	  N = 32  	  N = 64
mu = 1.0e+00	3.752e-02	1.877e-02	9.383e-03	4.692e-03
mu = 1.0e-01	7.671e-01	3.981e-01	2.010e-01	1.008e-01
mu = 1.0e-02	7.238e+00	6.684e+00	5.007e+00	2.969e+00


	    L2 error for P1 elements

           	  N = 8  	  N = 16 	  N = 32  	  N = 64
mu = 1.0e+00	1.402e-03	3.508e-04	8.770e-05	2.193e-05
mu = 1.0e-01	2.375e-02	6.177e-03	1.561e-03	3.915e-04
mu = 1.0e-02	2.379e-01	1.039e-01	3.819e-02	1.126e-02


	    H1 error for P2 elements

           	  N = 8  	  N = 16 	  N = 32  	  N = 64
mu = 1.0e+00	5.970e-04	1.504e-04	3.772e-05	9.448e-06
mu = 1.0e-01	1.183e-01	3.164e-02	8.066e-03	2.028e-03
mu = 1.0e-02	5.140e+00	3.604e+00	1.705e+00	5.661e-01


	    L2 error for P2 elements

           	  N = 8  	  N = 16 	  N = 32  	  N = 64
mu = 1.0e+00	1.151e-05	1.448e-06	1.817e-07	2.277e-08
mu = 1.0e-01	2.245e-03	3.038e-04	3.884e-05	4.888e-06
mu = 1.0e-02	8.513e-02	3.039e-02	7.598e-03	1.326e-03
\end{lstlisting}

\subsection*{c)}
The following program checks the validity of the error estimates (note that only the three first values for $\mu$ are used as only these give results)

\begin{lstlisting} [style=python, basicstyle= \tiny]
from dolfin import *
import numpy as np
import matplotlib.pyplot as plt
set_log_active(False)

muvals = [1., 0.1, 0.01]
nvals = [8,16,32,64]

for mu in muvals:
	for err_type in ['H1', 'L2']:
		for i in [1,2]:
		
			x = []
			y = []
			Alist = []
			for N in nvals:

				mesh = UnitSquareMesh(N,N)

				h = 1.0/N

				V = FunctionSpace(mesh, 'Lagrange', i)
				V_1 = FunctionSpace(mesh, 'Lagrange', i+2)

				u = TrialFunction(V)
				v = TestFunction(V)

				bc0 = DirichletBC(V, Constant(0.0), 'x[0] < DOLFIN_EPS')
				bc1 = DirichletBC(V, Constant(1.0), 'x[0] > 1-DOLFIN_EPS')
				bcs = [bc0, bc1]


				f = Constant(0.0)

				a = mu*inner(grad(u),grad(v))*dx + u.dx(0)*v*dx
				L = f*v*dx

				u_ = Function(V)

				solve(a == L, u_, bcs)

				
				u_ex = interpolate(Expression('(exp(x[0]/mu) - \
					1)/(exp(1./mu) - 1)', \
					mu=mu),V_1)

				A = errornorm(u_ex, u_, err_type, degree_rise=3)

				Alist.append(A)
				x.append(ln(h))
				y.append(ln(A))


			Alist = np.array(Alist)
			x = np.array(x)
			y = np.array(y)

			Arr = np.vstack([x, np.ones(len(x))]).T
			alpha, lnC = np.linalg.lstsq(Arr, y)[0]
			print('%s, mu = %.3e, P%d elements:\n' % (err_type, mu, i))
			print('alpha/beta = %.3f, C = %9.3f' % (alpha, exp(lnC)))
			#print('alpha/beta = %.3f, C = %9.3f' % (alpha, lnC))

			counter = 0
			for N in nvals:
				h = 1./N
				RHS = exp(lnC)*h**alpha
				print('N = %d: Errornorm = %.3e, Ch^alpha/beta = %.3e' \
					% (N, Alist[counter], RHS))
				if Alist[counter] < RHS:
					print('Error estimate is valid for N = %d!\n' % N)
				else:
					print('Error estimate is NOT valid for N = %d!\n' % N)
				counter += 1
			print('\n')
\end{lstlisting}
\
Below is a printout of the validity of the error estimates
\begin{lstlisting} [style=terminal, basicstyle= \tiny]
H1, mu = 1.000e+00, P1 elements:

alpha/beta = 1.000, C =     0.300

N = 8: Errornorm = 3.752e-02, Ch^alpha/beta = 3.752e-02
Error estimate is valid for N = 8!

N = 16: Errornorm = 1.877e-02, Ch^alpha/beta = 1.876e-02
Error estimate is NOT valid for N = 16!

N = 32: Errornorm = 9.383e-03, Ch^alpha/beta = 9.383e-03
Error estimate is NOT valid for N = 32!

N = 64: Errornorm = 4.692e-03, Ch^alpha/beta = 4.692e-03
Error estimate is valid for N = 64!



H1, mu = 1.000e+00, P2 elements:

alpha/beta = 1.994, C =     0.038

N = 8: Errornorm = 5.970e-04, Ch^alpha/beta = 5.979e-04
Error estimate is valid for N = 8!

N = 16: Errornorm = 1.504e-04, Ch^alpha/beta = 1.501e-04
Error estimate is NOT valid for N = 16!

N = 32: Errornorm = 3.772e-05, Ch^alpha/beta = 3.768e-05
Error estimate is NOT valid for N = 32!

N = 64: Errornorm = 9.448e-06, Ch^alpha/beta = 9.460e-06
Error estimate is valid for N = 64!



L2, mu = 1.000e+00, P1 elements:

alpha/beta = 2.000, C =     0.090

N = 8: Errornorm = 1.402e-03, Ch^alpha/beta = 1.403e-03
Error estimate is valid for N = 8!

N = 16: Errornorm = 3.508e-04, Ch^alpha/beta = 3.507e-04
Error estimate is NOT valid for N = 16!

N = 32: Errornorm = 8.770e-05, Ch^alpha/beta = 8.769e-05
Error estimate is NOT valid for N = 32!

N = 64: Errornorm = 2.193e-05, Ch^alpha/beta = 2.193e-05
Error estimate is valid for N = 64!

\end{lstlisting}
\newpage
\begin{lstlisting} [style=terminal, basicstyle= \tiny]
L2, mu = 1.000e+00, P2 elements:

alpha/beta = 2.994, C =     0.006

N = 8: Errornorm = 1.151e-05, Ch^alpha/beta = 1.152e-05
Error estimate is valid for N = 8!

N = 16: Errornorm = 1.448e-06, Ch^alpha/beta = 1.447e-06
Error estimate is NOT valid for N = 16!

N = 32: Errornorm = 1.817e-07, Ch^alpha/beta = 1.816e-07
Error estimate is NOT valid for N = 32!

N = 64: Errornorm = 2.277e-08, Ch^alpha/beta = 2.279e-08
Error estimate is valid for N = 64!



H1, mu = 1.000e-01, P1 elements:

alpha/beta = 0.977, C =     5.907

N = 8: Errornorm = 7.671e-01, Ch^alpha/beta = 7.745e-01
Error estimate is valid for N = 8!

N = 16: Errornorm = 3.981e-01, Ch^alpha/beta = 3.935e-01
Error estimate is NOT valid for N = 16!

N = 32: Errornorm = 2.010e-01, Ch^alpha/beta = 1.999e-01
Error estimate is NOT valid for N = 32!

N = 64: Errornorm = 1.008e-01, Ch^alpha/beta = 1.016e-01
Error estimate is valid for N = 64!



H1, mu = 1.000e-01, P2 elements:

alpha/beta = 1.957, C =     7.045

N = 8: Errornorm = 1.183e-01, Ch^alpha/beta = 1.204e-01
Error estimate is valid for N = 8!

N = 16: Errornorm = 3.164e-02, Ch^alpha/beta = 3.100e-02
Error estimate is NOT valid for N = 16!

N = 32: Errornorm = 8.066e-03, Ch^alpha/beta = 7.984e-03
Error estimate is NOT valid for N = 32!

N = 64: Errornorm = 2.028e-03, Ch^alpha/beta = 2.056e-03
Error estimate is valid for N = 64!



L2, mu = 1.000e-01, P1 elements:

alpha/beta = 1.975, C =     1.459

N = 8: Errornorm = 2.375e-02, Ch^alpha/beta = 2.400e-02
Error estimate is valid for N = 8!

N = 16: Errornorm = 6.177e-03, Ch^alpha/beta = 6.102e-03
Error estimate is NOT valid for N = 16!

N = 32: Errornorm = 1.561e-03, Ch^alpha/beta = 1.552e-03
Error estimate is NOT valid for N = 32!

N = 64: Errornorm = 3.915e-04, Ch^alpha/beta = 3.947e-04
Error estimate is valid for N = 64!



L2, mu = 1.000e-01, P2 elements:

alpha/beta = 2.950, C =     1.056

N = 8: Errornorm = 2.245e-03, Ch^alpha/beta = 2.291e-03
Error estimate is valid for N = 8!

N = 16: Errornorm = 3.038e-04, Ch^alpha/beta = 2.965e-04
Error estimate is NOT valid for N = 16!

N = 32: Errornorm = 3.884e-05, Ch^alpha/beta = 3.838e-05
Error estimate is NOT valid for N = 32!

N = 64: Errornorm = 4.888e-06, Ch^alpha/beta = 4.967e-06
Error estimate is valid for N = 64!

\end{lstlisting}
\newpage
\begin{lstlisting} [style=terminal, basicstyle= \tiny]
H1, mu = 1.000e-02, P1 elements:

alpha/beta = 0.427, C =    19.638

N = 8: Errornorm = 7.238e+00, Ch^alpha/beta = 8.076e+00
Error estimate is valid for N = 8!

N = 16: Errornorm = 6.684e+00, Ch^alpha/beta = 6.006e+00
Error estimate is NOT valid for N = 16!

N = 32: Errornorm = 5.007e+00, Ch^alpha/beta = 4.466e+00
Error estimate is NOT valid for N = 32!

N = 64: Errornorm = 2.969e+00, Ch^alpha/beta = 3.321e+00
Error estimate is valid for N = 64!



H1, mu = 1.000e-02, P2 elements:

alpha/beta = 1.063, C =    56.604

N = 8: Errornorm = 5.140e+00, Ch^alpha/beta = 6.209e+00
Error estimate is valid for N = 8!

N = 16: Errornorm = 3.604e+00, Ch^alpha/beta = 2.972e+00
Error estimate is NOT valid for N = 16!

N = 32: Errornorm = 1.705e+00, Ch^alpha/beta = 1.423e+00
Error estimate is NOT valid for N = 32!

N = 64: Errornorm = 5.661e-01, Ch^alpha/beta = 6.811e-01
Error estimate is valid for N = 64!



L2, mu = 1.000e-02, P1 elements:

alpha/beta = 1.465, C =     5.508

N = 8: Errornorm = 2.379e-01, Ch^alpha/beta = 2.619e-01
Error estimate is valid for N = 8!

N = 16: Errornorm = 1.039e-01, Ch^alpha/beta = 9.487e-02
Error estimate is NOT valid for N = 16!

N = 32: Errornorm = 3.819e-02, Ch^alpha/beta = 3.437e-02
Error estimate is NOT valid for N = 32!

N = 64: Errornorm = 1.126e-02, Ch^alpha/beta = 1.245e-02
Error estimate is valid for N = 64!



L2, mu = 1.000e-02, P2 elements:

alpha/beta = 2.001, C =     6.533

N = 8: Errornorm = 8.513e-02, Ch^alpha/beta = 1.018e-01
Error estimate is valid for N = 8!

N = 16: Errornorm = 3.039e-02, Ch^alpha/beta = 2.542e-02
Error estimate is NOT valid for N = 16!

N = 32: Errornorm = 7.598e-03, Ch^alpha/beta = 6.350e-03
Error estimate is NOT valid for N = 32!

N = 64: Errornorm = 1.326e-03, Ch^alpha/beta = 1.586e-03
Error estimate is valid for N = 64!
\end{lstlisting}

\newpage

\subsection*{d)}

To implement the SUPG method we replace the test function $v$ with a new test function $w = v + \beta u_x$. Inserted into the weak form for our problem, we get\\
\\
$\displaystyle \mu \int_{\Omega}\nabla u \nabla w \diff x + \int_{\Omega}u_x w \diff x = \mu \int_{\Omega}\nabla u \nabla ( v + \beta u_x) \diff x + \int_{\Omega}u_x ( v + \beta u_x) \diff x = $\\
\\
\\
$\displaystyle \mu \int_{\Omega}\nabla u \nabla v \diff x + \beta \mu \int_{\Omega}\nabla u \nabla u_x \diff x + \int_{\Omega}u_x  v  \diff x + \int_{\Omega}u_x^2 \diff x = 0$\\
\\
\\
Below is a printout of the computed values for $\alpha$ and C, as well as a error estimate validity check for each case
\begin{lstlisting} [style=terminal, basicstyle= \tiny]
mu = 1.000e+00, P1 elements:

alpha/beta = 1.131, C =     0.254

N = 8: Errornorm = 2.472e-02, Ch^alpha/beta = 2.419e-02
Error estimate is NOT valid for N = 8!

N = 16: Errornorm = 1.080e-02, Ch^alpha/beta = 1.105e-02
Error estimate is valid for N = 16!

N = 32: Errornorm = 4.948e-03, Ch^alpha/beta = 5.044e-03
Error estimate is valid for N = 32!

N = 64: Errornorm = 2.351e-03, Ch^alpha/beta = 2.303e-03
Error estimate is NOT valid for N = 64!



mu = 1.000e+00, P2 elements:

alpha/beta = 0.475, C =     2.378

N = 8: Errornorm = 8.799e-01, Ch^alpha/beta = 8.849e-01
Error estimate is valid for N = 8!

N = 16: Errornorm = 6.405e-01, Ch^alpha/beta = 6.365e-01
Error estimate is NOT valid for N = 16!

N = 32: Errornorm = 4.598e-01, Ch^alpha/beta = 4.578e-01
Error estimate is NOT valid for N = 32!

N = 64: Errornorm = 3.276e-01, Ch^alpha/beta = 3.293e-01
Error estimate is valid for N = 64!



mu = 1.000e-01, P1 elements:

alpha/beta = 1.062, C =     3.528

N = 8: Errornorm = 3.836e-01, Ch^alpha/beta = 3.878e-01
Error estimate is valid for N = 8!

N = 16: Errornorm = 1.886e-01, Ch^alpha/beta = 1.858e-01
Error estimate is NOT valid for N = 16!

N = 32: Errornorm = 8.921e-02, Ch^alpha/beta = 8.899e-02
Error estimate is NOT valid for N = 32!

N = 64: Errornorm = 4.234e-02, Ch^alpha/beta = 4.263e-02
Error estimate is valid for N = 64!

\end{lstlisting}
\newpage
\begin{lstlisting} [style=terminal, basicstyle= \tiny]

mu = 1.000e-01, P2 elements:

alpha/beta = 0.276, C =     1.806

N = 8: Errornorm = 9.853e-01, Ch^alpha/beta = 1.018e+00
Error estimate is valid for N = 8!

N = 16: Errornorm = 8.683e-01, Ch^alpha/beta = 8.410e-01
Error estimate is NOT valid for N = 16!

N = 32: Errornorm = 7.187e-01, Ch^alpha/beta = 6.947e-01
Error estimate is NOT valid for N = 32!

N = 64: Errornorm = 5.552e-01, Ch^alpha/beta = 5.739e-01
Error estimate is valid for N = 64!



mu = 1.000e-02, P1 elements:

alpha/beta = 0.395, C =     2.764

N = 8: Errornorm = 1.083e+00, Ch^alpha/beta = 1.215e+00
Error estimate is valid for N = 8!

N = 16: Errornorm = 1.031e+00, Ch^alpha/beta = 9.235e-01
Error estimate is NOT valid for N = 16!

N = 32: Errornorm = 7.957e-01, Ch^alpha/beta = 7.022e-01
Error estimate is NOT valid for N = 32!

N = 64: Errornorm = 4.735e-01, Ch^alpha/beta = 5.339e-01
Error estimate is valid for N = 64!



mu = 1.000e-02, P2 elements:

alpha/beta = 0.247, C =     2.859

N = 8: Errornorm = 1.681e+00, Ch^alpha/beta = 1.710e+00
Error estimate is valid for N = 8!

N = 16: Errornorm = 1.487e+00, Ch^alpha/beta = 1.441e+00
Error estimate is NOT valid for N = 16!

N = 32: Errornorm = 1.202e+00, Ch^alpha/beta = 1.214e+00
Error estimate is valid for N = 32!

N = 64: Errornorm = 1.019e+00, Ch^alpha/beta = 1.023e+00
Error estimate is valid for N = 64!

\end{lstlisting}
\end{document}